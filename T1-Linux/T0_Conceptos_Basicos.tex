\documentclass{beamer}
%\usepackage{pgf,pgfarrows,pgfnodes,pgfautomata,pgfheaps,pgfshade}
\usefonttheme{professionalfonts} % Font de LaTeX
\usetheme{Boadilla}
\usepackage{amsmath,amssymb}
%\usepackage[latin1]{inputenc}
\usepackage[utf8]{inputenc}
%\usepackage[spanish]{babel}
\usepackage{graphicx,verbatim,fancyvrb,array,listings,lastpage,eurosym}
\graphicspath{{/usr/local/texmf/logos/}{../imgs/}{FIGURES/png/}}
\usepackage{tikz}
%\usepackage[linesnumbered,algonl,boxed]{algorithm2e}
\usepackage[algonl,noend]{algorithm2e}
\usepackage{algpseudocode}
%%%%%%%%%%%%%% Tipo de letra %%%%%%%%%%%%%%%%%%%%%%%%%%%%%
%\usepackage{comicsans}
%\renewcommand{\sfdefault}{comic}
%%\usepackage{helvet}
%%\renewcommand{\familydefault}{\sfdefault}
%%%%%%%%%%%%%%%%%%%%%%%%%%%%%%%%%%%%%%%%%%%%%%%%%%%%%%%%%%
\usetikzlibrary{snakes,arrows,shapes}
\usetikzlibrary{automata}
%%%%%%%%%%%%%%%%%%%%%%%%%%%%%%%%%%%%%%%%%%%%%%%%%%%%%%%%%%%%%%%%%%%%%%%%%%%%%%%%%%%%%%%%%%%%
\newtheorem{teor}{Teorema}
\newtheorem{coro}{Corolario}
\newtheorem{demo}{Demostración}
\newtheorem{ejem}{Ejemplo}
\newtheorem{defi}{Definición}
\newtheorem{lema}{Lema}
%%%%%%%%%%%%%%%%%%%%%%%%%%%%%%%%%%%%%%%%%%%%%%%%%%%%%%%%%%%%%%%%%%%%%%%%%%%%%%%%%%%%%%%%%%%%
\definecolor{marron}     {rgb}{0.496, 0.203, 0.152}
\definecolor{verde-claro}{rgb}{0.600, 0.980, 0.600}
\definecolor{oscuro}     {rgb}{0.187, 0.141, 0.285}
\definecolor{gris}     	 {rgb}{0.500, 0.500, 0.500}
\definecolor{pastel1}     {rgb}{1.000, 1.000, 0.800}
\definecolor{pastel}    {rgb}{0.990, 0.960, 0.900}
%%%%%%%%%%%%%%%%%%%%%%%%%%%%%%%%%%%%%%%%%%%%%%%%%%%%%%%%%%%%%%%%%%%%%%%%%%%%%%%%%%%%%%%%%%%%
\usepackage{listings}
\lstloadlanguages{C,[90]Fortran}
\lstset{
 language=C,                              % C
 language=[90]Fortran,                    % Fortran90
 backgroundcolor=\color{pastel},         % Códigos sobre fondo amarillo
 %frame=lines,                            % Línea arriba y abajo de cada listado de código
 basicstyle=\small,                       % Tamaño de letra en listados
 captionpos=b,
 emph={pragma,omp,parallel},              % BGD: Resaltar las palabras pragma y omp 
 extendedchars=true,                      % BGD: Utiliza caracteres extendidos (tildes, etc.) 
 tabsize=2,                               % BGD: Tamaño de tabulador para indentaciones = 2
 breaklines=true,                         % BGD: Cortar líneas si son muy grandes
 breakautoindent=true,                    % BGD: Autoindentar líneas cortadas
 keywordstyle=\color{black}\textbf,       % BGD: Palabras clave en verde y negrita
 identifierstyle=\color{black}\ttfamily,
 commentstyle=\color{blue}\ttfamily,      % comentarios en marron
 stringstyle=\color{marron},              % cadenas en verde
 directivestyle=\color{black}\textbf,     % directivas
 showstringspaces=false
}
%%%%%%%%%%%%%%%%%%%%%%%%%%%%%%%%%%%%%%%%%%%%%%%%%%%%%%%%%%%%%%%%%%%%%%%%%%%%%%%%%%%%%%%%%%%%
\title{Conceptos Básicos}
%\author{F. de Sande}
%\date{2011--2012}
\begin{document}
\maketitle
%%%%%%%%%%%%%%%%%%%%%%%%%%%%%%%%%%%%%%%%%%%%%%%%%%%%%%%%%%%%%%%%%%%%%%%%%%%%%%%%%
%%%%%%%%%%%%%%%%%%%%%%%%%%%%%%%%%%%%%%%%%%%%%%%%%%%%%%%%%%%%%%%%%%%%%%%%%%%%%%%%%
\begin{frame}
  \frametitle{Alfabetos y Cadenas}
      \begin{block}{}
%%%%%%%%%%%%%%%%%%%%%%%%%%%%%%%%%%%%%%%%%
Consideremos los siguientes elementos:
\pause
\begin{itemize}[<+->]
\item Programas escritos en algún lenguaje de programación
\item Palabras en Euskera
\item Secuencias de símbolos que representan un valor entero
\item Frases escritas en castellano
\end{itemize}
      \end{block}
			\pause
      \begin{block}{Todos ellos tienen al menos dos elementos en común:}
\pause
\begin{itemize}[<+->]
\item Están compuestos por secuencias de símbolos tomados de un cojunto finito
\item Las secuencias de símbolos tienen una longitud finita
\end{itemize}
      \end{block}

\pause
\begin{defi}
\textbf{Alfabeto, $\Sigma$:}
Conjunto finito, no vacío de símbolos
\end{defi}
\end{frame}
%%%%%%%%%%%%%%%%%%%%%%%%%%%%%%%%%%%%%%%%%%%%%%%%%%%%%%%%%%%%%%%%%%%%%%%%%%%%%%%%%

%%%%%%%%%%%%%%%%%%%%%%%%%%%%%%%%%%%%%%%%%%%%%%%%%%%%%%%%%%%%%%%%%%%%%%%%%%%%%%%%%
\begin{frame}
  \frametitle{Alfabetos y Cadenas}
%%%%%%%%%%%%%%%%%%%%%%%%%%%%%%%%%%%%%%%%%
      \begin{block}{Ejemplos de alfabetos}
\begin{itemize}[<+->]
\item $\Sigma_1 = \{0, 1\}$ 
\item $\Sigma_2 = \{., -\}$
\item $\Sigma_3 = \{a, b, c, ..., z\}$
\item $\Sigma_4 = \{0, 1, 2, \ldots, 9\}$
\item $\Sigma_5 = \{\spadesuit, \heartsuit, \diamondsuit, \clubsuit\}$
\end{itemize}
		\end{block}
\end{frame}
%%%%%%%%%%%%%%%%%%%%%%%%%%%%%%%%%%%%%%%%%%%%%%%%%%%%%%%%%%%%%%%%%%%%%%%%%%%%%%%%%

%%%%%%%%%%%%%%%%%%%%%%%%%%%%%%%%%%%%%%%%%%%%%%%%%%%%%%%%%%%%%%%%%%%%%%%%%%%%%%%%%
\begin{frame}
  \frametitle{Alfabetos y Cadenas}
     \begin{defi}
%%%%%%%%%%%%%%%%%%%%%%%%%%%%%%%%%%%%%%%%%
\textbf{Cadena, palabra o frase:}
Secuencia finita de símbolos tomados de un alfabeto
\end{defi}

\pause 
\begin{block}{Por ejemplo:}
\begin{itemize}[<+->]
\item $w_1 = 0110$
\item $w_2 = ...---...$
\item $w_3 = abracadabra$ 
\item $w_4 = 31416$ 
\item $w_5 = \heartsuit \heartsuit \clubsuit \spadesuit \spadesuit \spadesuit$
\end{itemize}
\end{block}

\pause
\begin{block}{En CyA, las cadenas \textbf{no} tienen significado}
\end{block}

\end{frame}
%%%%%%%%%%%%%%%%%%%%%%%%%%%%%%%%%%%%%%%%%%%%%%%%%%%%%%%%%%%%%%%%%%%%%%%%%%%%%%%%%

%%%%%%%%%%%%%%%%%%%%%%%%%%%%%%%%%%%%%%%%%%%%%%%%%%%%%%%%%%%%%%%%%%%%%%%%%%%%%%%%%
\begin{frame}
  \frametitle{Alfabetos y Cadenas}
\begin{itemize}[<+->]
\item Dos cadenas con los mismos símbolos en diferente orden, son distintas: 
Sea $\Sigma = \{a, b, c\}, aca \neq caa$

\item El número de símbolos que componen la cadena es su longitud, y se denota $|w|$

\item La cadena vacía $\epsilon$ es la que no tiene ningún símbolo: $|\epsilon| = 0$

\item $\epsilon$ es una cadena sobre cualquier alfabeto $\Sigma$, puesto que $\epsilon$ es
una secuencia vacía de símbolos tomados de cualquier alfabeto
\end{itemize}
\end{frame}
%%%%%%%%%%%%%%%%%%%%%%%%%%%%%%%%%%%%%%%%%%%%%%%%%%%%%%%%%%%%%%%%%%%%%%%%%%%%%%%%%

%%%%%%%%%%%%%%%%%%%%%%%%%%%%%%%%%%%%%%%%%%%%%%%%%%%%%%%%%%%%%%%%%%%%%%%%%%%%%%%%%
\begin{frame}
  \frametitle{Lenguajes}
      \begin{defi}
			Un \textbf{lenguaje} (formal) es un conjunto de cadenas
			\end{defi}
%%%%%%%%%%%%%%%%%%%%%%%%%%%%%%%%%%%%%%%%%

\begin{block}{Ejemplos:}
\begin{itemize}[<+->]
\item $L_1 = \{1, 234, 912, 456\}$ es un lenguaje sobre $\Sigma = \{0, 1, 2, \ldots, 9\}$
\item El conjunto de palabras correctas en castellano es un lenguaje sobre el alfabeto latino
\item El conjunto de programas correctos escritos en C
\end{itemize}
\end{block}

\begin{itemize}[<+->]
\item Si $\Sigma$ es un alfabeto, también es un lenguaje
\item Los lenguajes pueden ser infinitos: $L = \{a, aa, aaa, aaaa, aaaaa, \ldots \}$
Este lenguaje es infinito, a pesar de que todas sus cadenas tienen longitud finita
\item Cuando el cardinal de un lenguaje es grande, resulta difícil especificar qué palabras lo componen
\item El lenguaje vacío, $L = \emptyset$ es un lenguaje
\end{itemize}
\end{frame}
%%%%%%%%%%%%%%%%%%%%%%%%%%%%%%%%%%%%%%%%%%%%%%%%%%%%%%%%%%%%%%%%%%%%%%%%%%%%%%%%%

%%%%%%%%%%%%%%%%%%%%%%%%%%%%%%%%%%%%%%%%%%%%%%%%%%%%%%%%%%%%%%%%%%%%%%%%%%%%%%%%%
\begin{frame}
  \frametitle{Lenguajes}
      \begin{block}{}
%%%%%%%%%%%%%%%%%%%%%%%%%%%%%%%%%%%%%%%%%
\begin{itemize}[<+->]
\item Estos lenguajes son distintos: $\emptyset \neq \{\epsilon\}$
\item Sea $\Sigma$ un alfabeto y $w$ una cadena sobre $\Sigma$
\item Si $L$ es un lenguaje formado por algunas de las cadenas sobre $\Sigma$ y $w$ está en $L$
entonces se denota $w \in L$ 
\item Por ejemplo, $241 \in \{123, 341, 241, 987\}$
\end{itemize}
			\end{block}
\end{frame}
%%%%%%%%%%%%%%%%%%%%%%%%%%%%%%%%%%%%%%%%%%%%%%%%%%%%%%%%%%%%%%%%%%%%%%%%%%%%%%%%%

%%%%%%%%%%%%%%%%%%%%%%%%%%%%%%%%%%%%%%%%%%%%%%%%%%%%%%%%%%%%%%%%%%%%%%%%%%%%%%%%%
\begin{frame}
  \frametitle{Lenguaje Universal, $\Sigma$}
      \begin{defi}
			\textbf{Lenguaje Universal}: es el lenguaje formado por todas las cadenas sobre un alfabeto $\Sigma$
			
			Se denota $\Sigma^*$
			\end{defi}

      %%%%%%%%%%%%%%%%%%%%%%%%%%%%%%%%%%%%%%%%%
      \pause
      \begin{block}{Ejemplos}
           \begin{itemize}[<+->]
           \item Si $\Sigma = \{x, 8\}$ entonces: $\Sigma^* = \{\epsilon, 8, x, xx, x8, 8x, 88, xxx, xx8, x8x, 8xx, \ldots\}$
           \item Si $\Sigma = \{\heartsuit\}$ entonces: 
					 $\Sigma^* = \{\epsilon, \heartsuit, \heartsuit \heartsuit, \heartsuit \heartsuit \heartsuit, \heartsuit \heartsuit \heartsuit \heartsuit, \ldots\}$
           \item Para cualquier alfabeto, $\Sigma^{*}$  es infinito (puesto que $\Sigma \not = \emptyset$)
           \end{itemize}
			\end{block}
      %%%%%%%%%%%%%%%%%%%%%%%%%%%%%%%%%%%%%%%%%
\end{frame}
%%%%%%%%%%%%%%%%%%%%%%%%%%%%%%%%%%%%%%%%%%%%%%%%%%%%%%%%%%%%%%%%%%%%%%%%%%%%%%%%%

%%%%%%%%%%%%%%%%%%%%%%%%%%%%%%%%%%%%%%%%%%%%%%%%%%%%%%%%%%%%%%%%%%%%%%%%%%%%%%%%%
\begin{frame}
  \frametitle{Operaciones con cadenas}
	   Sean $w, z \in \Sigma^*$
      \begin{defi}
			La concatenación $w \cdot z = wz$ es el resultado de yuxtaponer a la cadena $w$ los símbolos de $z$
      \end{defi}

      \pause
      \begin{block}{Ejemplo:}
      %%%%%%%%%%%%%%%%%%%%%%%%%%%%%%%%%%%%%%%%%
           \begin{itemize}[<+->]
           \item Sea $x =abra$, y $r = cadabra$
           \item Entonces $xr = abracadabra$
           \end{itemize}
      \end{block}
      %%%%%%%%%%%%%%%%%%%%%%%%%%%%%%%%%%%%%%%%%

      \pause
			\begin{block}{}
      %%%%%%%%%%%%%%%%%%%%%%%%%%%%%%%%%%%%%%%%%
           \begin{itemize}[<+->]
           \item $|wz| = |w| + |z|$
           \item $\epsilon$ es la identidad para la concatenación: $\forall w \in \Sigma^*, \epsilon w = w \epsilon = w$
           \end{itemize}
			\end{block}
      %%%%%%%%%%%%%%%%%%%%%%%%%%%%%%%%%%%%%%%%%

\end{frame}
%%%%%%%%%%%%%%%%%%%%%%%%%%%%%%%%%%%%%%%%%%%%%%%%%%%%%%%%%%%%%%%%%%%%%%%%%%%%%%%%%

%%%%%%%%%%%%%%%%%%%%%%%%%%%%%%%%%%%%%%%%%%%%%%%%%%%%%%%%%%%%%%%%%%%%%%%%%%%%%%%%%
\begin{frame}
  \frametitle{Operaciones con cadenas}
	   Sea $w \in \Sigma^*$, sea $n \in \mathbb{N}$
      \begin{defi}
                   \begin{displaymath}
                   w^n = \left\{ \begin{array}{ll}
                                  \epsilon & \textrm{si $n = 0$} \\
                                  ww^{n-1} & \textrm{si $n > 0$} \\
                                  \end{array} \right.
                   \end{displaymath}
      \end{defi}

      \pause
			\begin{block}{Ejemplo}
			  Sea $\Sigma=\{0, 1\}$ y $w=101$
      %%%%%%%%%%%%%%%%%%%%%%%%%%%%%%%%%%%%%%%%%
           \begin{itemize}[<+->]
           \item $w^0 = \epsilon$
           \item $w^1 = w = 101$
           \item $w^2 = ww = 101101$
           \item $w^3 = www = 101101101$
           \item $\ldots$
           \end{itemize}
			\end{block}
      %%%%%%%%%%%%%%%%%%%%%%%%%%%%%%%%%%%%%%%%%
\end{frame}
%%%%%%%%%%%%%%%%%%%%%%%%%%%%%%%%%%%%%%%%%%%%%%%%%%%%%%%%%%%%%%%%%%%%%%%%%%%%%%%%%

%%%%%%%%%%%%%%%%%%%%%%%%%%%%%%%%%%%%%%%%%%%%%%%%%%%%%%%%%%%%%%%%%%%%%%%%%%%%%%%%%
\begin{frame}
  \frametitle{Operaciones con cadenas}
	   Sean $w, z \in \Sigma^*$
      \begin{defi}
			Dos cadenas $w$ y $z$ son iguales si $|w|=|z|$ y tienen los mismos símbolos en las mismas posiciones
      \end{defi}

      \pause
			\begin{block}{Ejemplo}
			  Sea $\Sigma = \{ \alpha, \beta\}$ 
      %%%%%%%%%%%%%%%%%%%%%%%%%%%%%%%%%%%%%%%%%
           \begin{itemize}[<+->]
           \item $\alpha \beta \beta = \alpha \beta \beta$
           \item $\beta \beta \alpha \not = \beta \alpha \beta$
           \end{itemize}
			\end{block}
      %%%%%%%%%%%%%%%%%%%%%%%%%%%%%%%%%%%%%%%%%
\end{frame}
%%%%%%%%%%%%%%%%%%%%%%%%%%%%%%%%%%%%%%%%%%%%%%%%%%%%%%%%%%%%%%%%%%%%%%%%%%%%%%%%%

%%%%%%%%%%%%%%%%%%%%%%%%%%%%%%%%%%%%%%%%%%%%%%%%%%%%%%%%%%%%%%%%%%%%%%%%%%%%%%%%%
\begin{frame}
  \frametitle{Sufijos y prefijos}
      %%%%%%%%%%%%%%%%%%%%%%%%%%%%%%%%%%%%%%%%%
        Sean $w, x \in \Sigma^*$ palabras
      \begin{defi}
		  $x$ es prefijo de $w$ si $\exists y \in \Sigma^* | w=xy$	
			\end{defi}

			\pause
			Si $w=922318178$, entonces $x=922$ es un prefijo

			\pause
      \begin{defi}
		  Prefijos propios de una cadena son los prefijos que no son iguales a la cadena
			\end{defi}

			\pause
			$\epsilon$ es un prefijo de cualquier cadena

\end{frame}
%%%%%%%%%%%%%%%%%%%%%%%%%%%%%%%%%%%%%%%%%%%%%%%%%%%%%%%%%%%%%%%%%%%%%%%%%%%%%%%%%

%%%%%%%%%%%%%%%%%%%%%%%%%%%%%%%%%%%%%%%%%%%%%%%%%%%%%%%%%%%%%%%%%%%%%%%%%%%%%%%%%
\begin{frame}
  \frametitle{Subcadenas}
      %%%%%%%%%%%%%%%%%%%%%%%%%%%%%%%%%%%%%%%%%
      Sean $x, y, z, w \in \Sigma^*$ palabras
      \begin{defi}
			$w$ es \textbf{subcadena} de $z$ si existen las cadenas $x$ e $y$ para las cuales $z = xwy$
      \end{defi}
      \pause

      \begin{block}{Ejemplo}
           \begin{itemize}[<+->]
           \item Sea $w = abc$. Subcadenas de $w$ son:
					 \item $\epsilon$
					 \item $a$
					 \item $b$
					 \item $c$
					 \item $ab$
					 \item $bc$
					 \item $abc$
					 \item $\ldots$
           \end{itemize}
			\end{block}
\end{frame}
%%%%%%%%%%%%%%%%%%%%%%%%%%%%%%%%%%%%%%%%%%%%%%%%%%%%%%%%%%%%%%%%%%%%%%%%%%%%%%%%%

%%%%%%%%%%%%%%%%%%%%%%%%%%%%%%%%%%%%%%%%%%%%%%%%%%%%%%%%%%%%%%%%%%%%%%%%%%%%%%%%%
\begin{frame}
  \frametitle{Subsecuencias}
      %%%%%%%%%%%%%%%%%%%%%%%%%%%%%%%%%%%%%%%%%
      \begin{block}{Definición: Sean $x, y \in \Sigma^*$}
			$y$ es una \textbf{subsecuencia} de $x$ si $y$ tiene símbolos de $x$ respetando su orden, pero no necesariamente contiguos
      \end{block}
      \pause

      \begin{block}{}
           \begin{itemize}[<+->]
           \item $x = x_1x_2 \ldots x_N$
					 \item $y = x_{i1} x_{i2} \ldots x_{ik}$
					 \item $1 \leq i1 \leq i2 \leq \ldots \leq ik \leq im$
           \end{itemize}
			\end{block}

      \pause
      \begin{block}{Ejemplo}
           \begin{itemize}[<+->]
           \item Sea $w = abracadabra$. Algunas subsecuencias de $w$ son:
					 \item $arda, rcdr, aaaa$
					 \item $\epsilon$ es subsecuencia de toda cadena
			     \item Toda subcadena es subsecuencia, pero el recíproco no es cierto
			     \item Ejercicio: ¿cuál es el número de subsecuencias de $x \in \Sigma^*$ si $|x|=n$?
           \end{itemize}
			\end{block}
\end{frame}
%%%%%%%%%%%%%%%%%%%%%%%%%%%%%%%%%%%%%%%%%%%%%%%%%%%%%%%%%%%%%%%%%%%%%%%%%%%%%%%%%

%%%%%%%%%%%%%%%%%%%%%%%%%%%%%%%%%%%%%%%%%%%%%%%%%%%%%%%%%%%%%%%%%%%%%%%%%%%%%%%%%
\begin{frame}
  \frametitle{Inversa}
      %%%%%%%%%%%%%%%%%%%%%%%%%%%%%%%%%%%%%%%%%
      \begin{defi}
			Dada una cadena $w\in \Sigma^*$ se define su inversa como:


                   \begin{displaymath}
                   w^i = \left\{ \begin{array}{ll}
                                  w & \textrm{si $w = \epsilon$} \\
                                  y^ia & \textrm{si $w=ay$, con $a \in \Sigma, y \in \Sigma^*$} \\
                                  \end{array} \right.
                   \end{displaymath}
      \end{defi}
      \pause


      %%%%%%%%%%%%%%%%%%%%%%%%%%%%%%%%%%%%%%%%%
      \begin{block}{Ejemplo. Sea $w = arroz$}
           \begin{itemize}[<+->]
           \item $w^i = (arroz)^i =$
           \item $= (rroz)^i a =$
           \item $= (roz)^i ra =$
           \item $= (oz)^i rra =$
           \item $= (z)^i orra =$
           \item $= (\epsilon)^i zorra =$
           \item $= \epsilon zorra =$
           \item $= zorra =$
           \end{itemize}
			\end{block}
      %%%%%%%%%%%%%%%%%%%%%%%%%%%%%%%%%%%%%%%%%
\end{frame}
%%%%%%%%%%%%%%%%%%%%%%%%%%%%%%%%%%%%%%%%%%%%%%%%%%%%%%%%%%%%%%%%%%%%%%%%%%%%%%%%%

%%%%%%%%%%%%%%%%%%%%%%%%%%%%%%%%%%%%%%%%%%%%%%%%%%%%%%%%%%%%%%%%%%%%%%%%%%%%%%%%%
\begin{frame}
  \frametitle{Operaciones con lenguajes}
      %%%%%%%%%%%%%%%%%%%%%%%%%%%%%%%%%%%%%%%%%
				Las operaciones de conjuntos se pueden extender a lenguajes (los lenguajes son conjuntos...)

        \pause
			  Sean $L_1$ y $L_2$ lenguajes sobre un alfabeto $\Sigma$
      \pause

      \begin{defi}
			\textbf{Concatenación} (producto cartesiano): 
			
			$L_1 \cdot L_2 = L_1 L_2 = \{xy | x \in L_1 \wedge y \in L_2\}$

			$L_1 L_2$ está formado por todas las cadenas formadas concatenando una cadena de $L_1$ con otra de $L_2$
			\end{defi}

      \pause
      \begin{block}{Si $L_1 = \{ ni$ñ@$, chic$@$ \}$ y  $L_2 = \{ buen$@$, mal$@$ \}$}
           \begin{itemize}[<+->]
           \item $L_1 L_2 = \{ ni$ñ@$buen$@$, ni$ñ@$mal$@$, chic$@$buen$@$, chic$@$mal$ @ $ \}$
           \item Si $L_1$ es un lenguaje sobre $\Sigma_1$ y $L_2$ es un lenguaje sobre $\Sigma_2$, $L_1 L_2$ es un lenguaje
					       sobre $\Sigma_1 \cup \Sigma_2$
           \end{itemize}
			\end{block}
      %%%%%%%%%%%%%%%%%%%%%%%%%%%%%%%%%%%%%%%%%
\end{frame}
%%%%%%%%%%%%%%%%%%%%%%%%%%%%%%%%%%%%%%%%%%%%%%%%%%%%%%%%%%%%%%%%%%%%%%%%%%%%%%%%%

%%%%%%%%%%%%%%%%%%%%%%%%%%%%%%%%%%%%%%%%%%%%%%%%%%%%%%%%%%%%%%%%%%%%%%%%%%%%%%%%%
\begin{frame}
  \frametitle{Operaciones con lenguajes}
      %%%%%%%%%%%%%%%%%%%%%%%%%%%%%%%%%%%%%%%%%
			  Sea $L$ un lenguaje sobre un alfabeto $\Sigma$
      \pause

      \begin{defi}
                   \begin{displaymath}
                   L^n = \left\{ \begin{array}{ll}
                                  \{\epsilon\} & \textrm{si $n = 0$} \\
                                  LL^{n-1} & \textrm{si $n > 0$} \\
                                  \end{array} \right.
                   \end{displaymath}
      \end{defi}
      \pause
      \begin{block}{Ejemplo. Si $L = \{ 0 , 1 \}$ }
           \begin{itemize}[<+->]
           \item $L^0 = \{ \epsilon \}$
           \item $L^1 = L =  \{ 0 , 1 \}$
           \item $L^2 = LL = \{ 00, 01, 10, 11 \}$
           \item $L^3 = LL^2 = \{ 000, 001, 010, 011, 100, 101, 110, 111 \}$
           \item $\ldots$
           \item Obsérvese que $\emptyset ^ 0 =  \{ \epsilon \}$
           \item $\emptyset ^ n = \emptyset \; \; \forall n \geq 1$
           \item $\emptyset ^ 1 = \emptyset \emptyset^0 = \{xy | x \in \emptyset \wedge y \in \{\epsilon\} \} = \emptyset$
           \end{itemize}
			\end{block}
      %%%%%%%%%%%%%%%%%%%%%%%%%%%%%%%%%%%%%%%%%
\end{frame}
%%%%%%%%%%%%%%%%%%%%%%%%%%%%%%%%%%%%%%%%%%%%%%%%%%%%%%%%%%%%%%%%%%%%%%%%%%%%%%%%%

%%%%%%%%%%%%%%%%%%%%%%%%%%%%%%%%%%%%%%%%%%%%%%%%%%%%%%%%%%%%%%%%%%%%%%%%%%%%%%%%%
\begin{frame}
  \frametitle{Operaciones con lenguajes}
      %%%%%%%%%%%%%%%%%%%%%%%%%%%%%%%%%%%%%%%%%
			  Sean $L_1$ y $L_2$ lenguajes sobre un alfabeto $\Sigma$
      \pause

      \begin{defi}
			\textbf{Unión}: $L_1 \cup L_2 = \{x | x \in L_1 \vee x \in L_2\}$
			\end{defi}
      \pause

      \begin{defi}
			\textbf{Intersección}: $L_1 \cap L_2 = \{x | x \in L_1 \wedge x \in L_2\}$
			\end{defi}
      \pause

      \begin{defi}
			\textbf{Sublenguaje}: $L_1$ es un sublenguaje de $L_2$ si $L_1 \subseteq L_2$
			\end{defi}
      \pause

      \begin{defi}
			\textbf{Igualdad}: $L_1 = L_2$ Si $L_1 \subseteq L_2$ y $L_2 \subseteq L_1$
			\end{defi}
      %%%%%%%%%%%%%%%%%%%%%%%%%%%%%%%%%%%%%%%%%
\end{frame}
%%%%%%%%%%%%%%%%%%%%%%%%%%%%%%%%%%%%%%%%%%%%%%%%%%%%%%%%%%%%%%%%%%%%%%%%%%%%%%%%%

%%%%%%%%%%%%%%%%%%%%%%%%%%%%%%%%%%%%%%%%%%%%%%%%%%%%%%%%%%%%%%%%%%%%%%%%%%%%%%%%%
\begin{frame}
  \frametitle{Operaciones con lenguajes}
      %%%%%%%%%%%%%%%%%%%%%%%%%%%%%%%%%%%%%%%%%
			Sean $L_1, L_2, L_3$ lenguajes sobre $\Sigma$
      \begin{teor}
           \begin{itemize}[<+->]
           \item $L_1(L_2 \cup L_3) = L_1 L_2 \cup L_1 L_3$
           \item $(L_2 \cup L_3) L_1 = L_2 L_1 \cup L_3 L_1$
           \end{itemize}
      \end{teor}
      %%%%%%%%%%%%%%%%%%%%%%%%%%%%%%%%%%%%%%%%%
\end{frame}
%%%%%%%%%%%%%%%%%%%%%%%%%%%%%%%%%%%%%%%%%%%%%%%%%%%%%%%%%%%%%%%%%%%%%%%%%%%%%%%%%

%%%%%%%%%%%%%%%%%%%%%%%%%%%%%%%%%%%%%%%%%%%%%%%%%%%%%%%%%%%%%%%%%%%%%%%%%%%%%%%%%
\begin{frame}
  \frametitle{Operaciones con lenguajes}
      %%%%%%%%%%%%%%%%%%%%%%%%%%%%%%%%%%%%%%%%%
			  Sea $L$ un lenguaje sobre un alfabeto $\Sigma$

      \pause
      \begin{defi}
			\textbf{Cierre de Kleene} (o cierre estrella) de $L$:

			$L ^* = \displaystyle\bigcup_{n=0}^{\infty} L^n$
      \end{defi}

			\pause
			Las cadenas de $L^*$ se forman realizando 0 ó más concatenaciones de cadenas de $L$

      \pause
      \begin{defi}
			\textbf{Cierre positivo} de $L$:

			$L ^+ = \displaystyle\bigcup_{n=1}^{\infty} L^n$
			\end{defi}

			 Las cadenas de $L^+$ se forman realizando 1 ó más concatenaciones de cadenas de $L$


\end{frame}
%%%%%%%%%%%%%%%%%%%%%%%%%%%%%%%%%%%%%%%%%%%%%%%%%%%%%%%%%%%%%%%%%%%%%%%%%%%%%%%%%

%%%%%%%%%%%%%%%%%%%%%%%%%%%%%%%%%%%%%%%%%%%%%%%%%%%%%%%%%%%%%%%%%%%%%%%%%%%%%%%%%
\begin{frame}
  \frametitle{Operaciones con lenguajes}
      %%%%%%%%%%%%%%%%%%%%%%%%%%%%%%%%%%%%%%%%%
      \begin{block}{Sea $L=\{ a \}$}
           \begin{itemize}[<+->]
           \item $L^0 = \{ \epsilon \} $ 
           \item $L^1 = \{ a \} $ 
           \item $L^2 = \{ aa \} $ 
           \item $L^3 = \{ aaa \} $ 
           \item $L^* = \{ \epsilon, a, aa, aaa, aaaa, \ldots \} $ 
           \item $L^+ = \{ a, aa, aaa, aaaa, \ldots \} $ 
           \end{itemize}
			\end{block}
      %%%%%%%%%%%%%%%%%%%%%%%%%%%%%%%%%%%%%%%%%

      %%%%%%%%%%%%%%%%%%%%%%%%%%%%%%%%%%%%%%%%%
      \pause
      \begin{block}{Si $\Sigma$ es un alfabeto}
           \begin{itemize}[<+->]
           \item $\Sigma^*$ son todas las cadenas sobre $\Sigma$ 
					 
					       (concatenación de 0 ó más símbolos de $\Sigma$)
           \item y por tanto la notación es coherente
           \end{itemize}
			\end{block}
      %%%%%%%%%%%%%%%%%%%%%%%%%%%%%%%%%%%%%%%%%
\end{frame}
%%%%%%%%%%%%%%%%%%%%%%%%%%%%%%%%%%%%%%%%%%%%%%%%%%%%%%%%%%%%%%%%%%%%%%%%%%%%%%%%%

%%%%%%%%%%%%%%%%%%%%%%%%%%%%%%%%%%%%%%%%%%%%%%%%%%%%%%%%%%%%%%%%%%%%%%%%%%%%%%%%%
\begin{frame}
  \frametitle{Operaciones con lenguajes}
      %%%%%%%%%%%%%%%%%%%%%%%%%%%%%%%%%%%%%%%%%
      \begin{block}{Propiedades de las operaciones de cierre}
           \begin{itemize}[<+->]
           \item Si $L$ es un lenguaje sobre $\Sigma$, $L \subseteq \Sigma^*$
           \item Si $L$ es un lenguaje sobre $\Sigma$, $L^n \subseteq \Sigma^* \; \; \forall n \in \mathbb{N}$
           \item $L^* \subseteq \Sigma^*$
           \item $L^+ \subseteq \Sigma^*$
           \item $\emptyset^0 = \{ \epsilon \}$ y $\emptyset^n = \emptyset \; \; \forall n \geq 1$
           \item Por lo tanto: $\emptyset ^* =  \{ \epsilon \}$ y $\emptyset ^+ = \emptyset$
           \end{itemize}
			\end{block}
      %%%%%%%%%%%%%%%%%%%%%%%%%%%%%%%%%%%%%%%%%

      \pause
			Sean $L_1, L_2 \subseteq \Sigma^*$
      \begin{defi}
			$L_1 - L_2 = \{w \in \Sigma^* | w \in L_1 \wedge w \not \in L_2\}$
      \end{defi}
      \pause

      \pause
			Sean $L \subseteq \Sigma^*$
      \begin{defi}
			\textbf{Complementario}: $\bar L = \Sigma^* - L$
      \end{defi}
\end{frame}
%%%%%%%%%%%%%%%%%%%%%%%%%%%%%%%%%%%%%%%%%%%%%%%%%%%%%%%%%%%%%%%%%%%%%%%%%%%%%%%%%

%%%%%%%%%%%%%%%%%%%%%%%%%%%%%%%%%%%%%%%%%%%%%%%%%%%%%%%%%%%%%%%%%%%%%%%%%%%%%%%%%
\begin{frame}
  \frametitle{Operaciones con Lenguajes}
      %%%%%%%%%%%%%%%%%%%%%%%%%%%%%%%%%%%%%%%%%
      \begin{teor}
			$\forall A \subseteq \Sigma^*: \; \; A^+ = AA^* = A^*A$
      \end{teor}
      \pause

      \begin{demo}
           \uncover<2-> {Sea $x \in A^+ \; \Rightarrow x \in \displaystyle\bigcup_{n=1}^{\infty} A^n \; \; \Rightarrow$}

           \uncover<3-> {$\Rightarrow \exists k \geq 1 | \; x \in A^k = AA^{k-1}  \Rightarrow$}

					 \uncover<4-> {$\Rightarrow x \in \displaystyle\bigcup_{n=0}^{\infty} (AA)^n \; \; = $}

					 \uncover<5-> {$= A \displaystyle\bigcup_{n=0}^{\infty} A^n \; \; = AA^*$ de modo que }

					 \uncover<6-> {$A^+ \subseteq AA^*$}
      \end{demo}
\end{frame}
%%%%%%%%%%%%%%%%%%%%%%%%%%%%%%%%%%%%%%%%%%%%%%%%%%%%%%%%%%%%%%%%%%%%%%%%%%%%%%%%%

%%%%%%%%%%%%%%%%%%%%%%%%%%%%%%%%%%%%%%%%%%%%%%%%%%%%%%%%%%%%%%%%%%%%%%%%%%%%%%%%%
\begin{frame}
  \frametitle{Operaciones con Lenguajes}
      %%%%%%%%%%%%%%%%%%%%%%%%%%%%%%%%%%%%%%%%%
      \begin{teor}
			$\forall A \subseteq \Sigma^*: \; \; A^+ = AA^* = A^*A$
      \end{teor}

      Veamos el inverso:
			\pause
      \begin{demo}
					 \uncover<2-> {Sea $x \in AA^* = A  \displaystyle\bigcup_{n=0}^{\infty} A^n =$}

					 \uncover<3-> {$= \displaystyle\bigcup_{n=0}^{\infty} (AA^n) \Rightarrow$}

					 \uncover<4-> {$\Rightarrow \exists j \geq 0 | \; x \in AA^j = A^{j+1} \subseteq  \displaystyle\bigcup_{n=1}^{\infty} A^k = A^+ \Rightarrow$}

					 \uncover<5-> {$\Rightarrow AA^* \subseteq A^+$}

					 \uncover<6-> {La demostración del otro teorema es similar a ésta}
      \end{demo}
\end{frame}
%%%%%%%%%%%%%%%%%%%%%%%%%%%%%%%%%%%%%%%%%%%%%%%%%%%%%%%%%%%%%%%%%%%%%%%%%%%%%%%%%

%%%%%%%%%%%%%%%%%%%%%%%%%%%%%%%%%%%%%%%%%%%%%%%%%%%%%%%%%%%%%%%%%%%%%%%%%%%%%%%%%
\begin{frame}
  \frametitle{Operaciones con Lenguajes}
      %%%%%%%%%%%%%%%%%%%%%%%%%%%%%%%%%%%%%%%%%
      \begin{teor}
           \begin{itemize}
					 \item $(A^+)^+ = A^+$
					 \item $(A^*)^* = A^*$
           \end{itemize}
      \pause De estas propiedades proviene el nombre de \textit{cierre} de estas operaciones: no se añaden
			       cadenas adicionales a $A^*$ (o a $A^+$) aunque se realicen nuevas operaciones de cierre

      \pause 
           \begin{itemize}[<+->]
					 \item $(L^+)^* = L^*$
					 \item $(L^*)^+ = L^*$
					 \item $L^+ = L^* L = L L^*$
					 \item $L \subseteq L^+ \subseteq L^*$
					 \item $L_1 \subseteq L_2 \Rightarrow L_1^n \subseteq L_2^n \; \; \forall n \in \mathbb{N}$
					 \item $L_1 \subseteq L_2 \Rightarrow L_1^* \subseteq L_2^* \; \; (L_1^+ \subseteq L_2^+)$
           \end{itemize}
      \end{teor}

\end{frame}
%%%%%%%%%%%%%%%%%%%%%%%%%%%%%%%%%%%%%%%%%%%%%%%%%%%%%%%%%%%%%%%%%%%%%%%%%%%%%%%%%

%%%%%%%%%%%%%%%%%%%%%%%%%%%%%%%%%%%%%%%%%%%%%%%%%%%%%%%%%%%%%%%%%%%%%%%%%%%%%%%%%
\begin{frame}
  \frametitle{Operaciones con Lenguajes}
      %%%%%%%%%%%%%%%%%%%%%%%%%%%%%%%%%%%%%%%%%
      Sea $L \subseteq \Sigma^*$
      \begin{defi}
			\textbf{Inverso} de $L$: $L^{-1} = \{w \in \Sigma^*| \; w^i \in L \}$
      \end{defi}

			\pause
      \begin{block}{Ejemplo: si $L=\{alfa, beta\}$}
			$L^{-1} = \{ afla, ateb \}$
      \end{block}
\end{frame}
%%%%%%%%%%%%%%%%%%%%%%%%%%%%%%%%%%%%%%%%%%%%%%%%%%%%%%%%%%%%%%%%%%%%%%%%%%%%%%%%%

%%%%%%%%%%%%%%%%%%%%%%%%%%%%%%%%%%%%%%%%%%%%%%%%%%%%%%%%%%%%%%%%%%%%%%%%%%%%%%%%%
\end{document}
