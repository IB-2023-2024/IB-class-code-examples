%%%%%%%%%%%%%%%%%%%%%%%%%%%%%%%%%%%%%%%%%%%%%%%%%%%%%%%%%%%%%%%%%%%%%%%%%%%%%%%%%
\begin{frame}
  \frametitle{Operaciones con lenguajes}
      %%%%%%%%%%%%%%%%%%%%%%%%%%%%%%%%%%%%%%%%%
				Las operaciones de conjuntos se pueden extender a lenguajes (los lenguajes son conjuntos...)

        \pause
			  Sean $L_1$ y $L_2$ lenguajes sobre un alfabeto $\Sigma$
      \pause

      \begin{defi}
			\textbf{Concatenación} (producto cartesiano): 
			
			$L_1 \cdot L_2 = L_1 L_2 = \{xy | x \in L_1 \wedge y \in L_2\}$

			$L_1 L_2$ está formado por todas las cadenas formadas concatenando una cadena de $L_1$ con otra de $L_2$
			\end{defi}

      \pause
      \begin{block}{Si $L_1 = \{ ni$ñ@$, chic$@$ \}$ y  $L_2 = \{ buen$@$, mal$@$ \}$}
           \begin{itemize}[<+->]
           \item $L_1 L_2 = \{ ni$ñ@$buen$@$, ni$ñ@$mal$@$, chic$@$buen$@$, chic$@$mal$ @ $ \}$
           \item Si $L_1$ es un lenguaje sobre $\Sigma_1$ y $L_2$ es un lenguaje sobre $\Sigma_2$, $L_1 L_2$ es un lenguaje
					       sobre $\Sigma_1 \cup \Sigma_2$
           \end{itemize}
			\end{block}
      %%%%%%%%%%%%%%%%%%%%%%%%%%%%%%%%%%%%%%%%%
\end{frame}
%%%%%%%%%%%%%%%%%%%%%%%%%%%%%%%%%%%%%%%%%%%%%%%%%%%%%%%%%%%%%%%%%%%%%%%%%%%%%%%%%
