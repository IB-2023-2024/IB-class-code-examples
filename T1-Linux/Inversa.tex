%%%%%%%%%%%%%%%%%%%%%%%%%%%%%%%%%%%%%%%%%%%%%%%%%%%%%%%%%%%%%%%%%%%%%%%%%%%%%%%%%
\begin{frame}
  \frametitle{Inversa}
      %%%%%%%%%%%%%%%%%%%%%%%%%%%%%%%%%%%%%%%%%
      \begin{defi}
			Dada una cadena $w\in \Sigma^*$ se define su inversa como:


                   \begin{displaymath}
                   w^i = \left\{ \begin{array}{ll}
                                  w & \textrm{si $w = \epsilon$} \\
                                  y^ia & \textrm{si $w=ay$, con $a \in \Sigma, y \in \Sigma^*$} \\
                                  \end{array} \right.
                   \end{displaymath}
      \end{defi}
      \pause


      %%%%%%%%%%%%%%%%%%%%%%%%%%%%%%%%%%%%%%%%%
      \begin{block}{Ejemplo. Sea $w = arroz$}
           \begin{itemize}[<+->]
           \item $w^i = (arroz)^i =$
           \item $= (rroz)^i a =$
           \item $= (roz)^i ra =$
           \item $= (oz)^i rra =$
           \item $= (z)^i orra =$
           \item $= (\epsilon)^i zorra =$
           \item $= \epsilon zorra =$
           \item $= zorra =$
           \end{itemize}
			\end{block}
      %%%%%%%%%%%%%%%%%%%%%%%%%%%%%%%%%%%%%%%%%
\end{frame}
%%%%%%%%%%%%%%%%%%%%%%%%%%%%%%%%%%%%%%%%%%%%%%%%%%%%%%%%%%%%%%%%%%%%%%%%%%%%%%%%%
